

\begin{abstract}
%Company runs systems that have been implemented a long time ago. These systems usually are still under adaptation and maintenance to address current needs. Very often, adapting legacy software systems to new requirements needs to make use of new technological advances. 
Legacy systems mainly consist of two kinds of artifacts: source code and databases. Usually, the maintenance of those artifacts is carried out through refactoring processes in isolated manners and without following any kind of standardization. In order to provide a more effective maintenance of the whole system both artifacts should be analyzed, evolved jointly and standardized way. As is known refactoring is a very useful process for dealing with software aging problem since it can improve maintainability and reusability of these system. Recently, researches have been shifted from the typical refactoring processes to the so-called Architecture-Driven Modernization (ADM), which is an approach that follows the all the principles of Model-Driven Development (MDD). As one of the challenges of Software Engineering focuses on mechanisms to support the automation of software refactoring process in this paper we put forward KnowDIME to assist the modernization of a legacy system based on ADM, which uses the Knowledge Discovery Metamodel (KDM) standard. This infrastructure analyses Structured Query Language (SQL) queries embedded in a legacy source code in order to restructure and re-organize the system by using design patterns, such as Data Access Object (DAO) and Service-Oriented Architecture (SOA). 
\end{abstract}
% IEEEtran.cls defaults to using nonbold math in the Abstract.
% This preserves the distinction between vectors and scalars. However,
% if the conference you are submitting to favors bold math in the abstract,
% then you can use LaTeX's standard command \boldmath at the very start
% of the abstract to achieve this. Many IEEE journals/conferences frown on
% math in the abstract anyway.

% no keywords




% For peer review papers, you can put extra information on the cover
% page as needed:
% \ifCLASSOPTIONpeerreview
% \begin{center} \bfseries EDICS Category: 3-BBND \end{center}
% \fi
%
% For peerreview papers, this IEEEtran command inserts a page break and
% creates the second title. It will be ignored for other modes.
\IEEEpeerreviewmaketitle

\section{Introduction}
	Companies have a vast number of operational legacy systems and these systems are not immune to software aging problems. However, they can not be discarded because they incorporate a lot of embodied knowledge due to years of maintenance and this constitutes a significant corporate asset. Furthermore, these kind of systems mainly consist of two artifacts: source code and databases. Usually, the maintenance of those artifacts is carried out through restructuring processes in isolated manners~\cite{Moser:2006}. Nevertheless, we claim that for a more effective maintenance of the whole system both should be analyzed and evolved jointly. 

Refactoring can be used in order to deal with the maintenance of these systems. Martin Fowler~\cite{refactImpro} defines refactoring as ``A change made to the internal structure of software to make it easier to understand and cheaper to modify without changing its observable behavior''. In the literature there is a lot of studies that make refactoring at the level of source code. Empirical studies of refactoring have shown that it can improve maintainability~\cite{1510132} and reusability~\cite{Moser:2006} of legacy system. Not only does existing work suggest that refactoring is useful, but it also suggests that refactoring is a frequent practice~\cite{Murphy:2011}. Cherubini and colleagues` survey indicates that developers rate the importance of refactoring as equal to or greater than that of understanding code and producing documentation~\cite{Cherubini:2007}.

In a parallel research line, researchers have been shifted from the typical refactoring process to the so-called Architecture-Driven Modernization (ADM)~\cite{Ulrich:2010}. ADM has been proposed by OMG (Object Management Group) with the concept of modernizing legacy systems with a focus on all aspects of the current system architecture and the ability to transform current architectures to target architectures. In addition, ADM advocates conducting reengineering processes following the principles of Model-Driven Development (MDD)~\cite{Ulrich:2010:IST:1841736}, i.e., it treats all the software artifacts involved in the legacy system as models and can establish transformations among them.  Firstly a reverse engineering is performed starting from the source code and a model instance (\textbf{P}lataform \textbf{S}pecific \textbf{M}odel - PSM) is created. Next successive refinements (transformations) are applied to this model up to reach a good abstraction level (PSM) in model called KDM (\textbf{K}nowledge \textbf{D}iscovery \textbf{M}etamodel). Upon this model, several refactorings, optimizations and modifications can be performed in order to solve problems found in the legacy system. Secondly a forward engineering is carried out and the source code of the modernized target system is generated again. According to the OMG the most important artifact provided by ADM is the KDM metamodel, which is a multipurpose standard metamodel that represents all aspects of the existing IT (Information  Technology) architectures. The idea behind the standard KDM is that the community starts to create parsers from different languages to KDM. As a result everything that takes KDM as input can be considered platform and language-independent.

Accordingly, one of the challenges of Software Engineering focuses on specifications, architectures and mechanisms to support the automation of software reengineering process, aiming to reduce time and effort spent in this process. Furthermore, legacy systems, often with high maintenance costs due to scarcity or absence of documentation, may be restructured and ported to ADM, providing greater quality in development and maintenance of these systems. Motivated by these ideas, we put forward an infrastructure based on the metamodel KDM which support legacy system reengineering from databases queries and source code, with the objectives of reducing the time and effort in this process and providing migration of these systems to ADM. More specifically, this infrastructure analyses Structured Query Language (SQL) queries embedded in a legacy source code in order to restructure and re-organize the system by using design patterns, such as Data Access Object (DAO) and Service-Oriented Architecture (SOA).  This paper is organized as followed: Section 2 provides information related to the infrastructure - Section 3 the architecture of the infrastructure is depicted - in Section 4 there are related works and in Section 5 we conclude the paper with some remarks and future directions.

	%This paper is structured as follows: in Section II, Crosscutting Frameworks are explained; in Section III, the Proposed Model and the Reuse Process are shown; in Section IV, a tool to support the process is used to reuse a persistence framework as an Example; in Section V, an empirical evaluation is presented; in Section VI, there are related works and in Section VII, there are the conclusions.
	
	
\section{Software Restructuring\label{software restructuring}}
	Perhaps the most common of all software engineering activities is the modifications of software. Unfortunately, software modification, i.e., software maintenance, often leaves behind software that is difficult to understand for those other than its author. In this context, software restructuring is a field that seeks to reverse these effects on software. 

More specifically, software restructuring is the modification of software to make the software easier to understand and to change, or less susceptible to error when future changes are made (ref). In other words, it is the process of re-organizing the logical structure of existing software system to improve specific attributes~\cite{kang1999}. Some examples of software restructuring are improving coding style, editing documentation, transforming program components (moving class, creating class, etc). The central idea of restructuring is the action of transformation. According to~\cite{Eloff2002} a transformation can be defined formally as a function that receives a program, P, as input and produces a new program, P'. Thus, P' is said to be functionally equivalent to P $\Leftrightarrow$ P' exhibits identical behavior  to P for all defined inputs of P. Finally, T is called a meaning preserving transformation if P' $\equiv$ P.

According to Griswold's experiments programmers tends to not only commit syntactic errors and behave inconsistently, they usually ignore the global impact of the changes they make 

Manually restructuring software may have undesirable, and often unforeseen results that can affect the be- haviour of a system. Griswold’s experiments found that programmers not only commit syntactic errors and behave inconsistently, but they also ignore the global impact of the changes they make [Griswold 1991]. Fur- thermore, manual techniques demand the maintainer to guarantee the preservation of the system’s behaviour.


\section{Model-Driven Restructuring\label{MDD}}
	According to Griswold's experiments programmers tends to not only commit syntactic errors an behave inconsistently, they usually ignore the global impact of the changes they make during the restructuring process. Moreover, he also argues that manual restructuring is an error-prone and expensive activity~\cite{grisswold}. Therefore, software engineers have applied Model-Driven Development (MDD) technologies to software restructuring to deal with those limitations and to automatize the software restructuring. MDD consists of the combination of generative programming, domain-specific languages and model transformations. It also aims to reduce the semantic gap between the program domain and the implementation, using high-level models that shield software developers from complexities of the underlying implementation platform~\cite{France:2007:MDC:1253532.1254709}.
		
%\section{Evaluation}
%	\input{tex/evaluation}
%	\subsection{Reuse Study Definition}
%		\input{tex/reuse_study_definition}
%	\subsection{Maintenance Study Definition}
%		\input{tex/maintenance_study_definition}
%	\subsection{Study Planning}
%		\input{tex/study_planning}
%	\subsection{Operation}
%		\input{tex/operation}
%	\subsection{Data Analysis and Interpretation}
%		\input{tex/data_analysis_and_interpretation}
%	\subsection{Hypotheses Testing}
%		\input{tex/hypotheses_testing}
%	\subsection{Threats to Validity}
%		\input{tex/threats_to_validity}
\section{Related Work}
		\input{tex/related_works}
\section{Conclusions}
		In this paper is presented the KnowDIME to support the refactoring of legacy systems based on ADM, which uses the KDM standard. It follows the theory of the horseshoe modernization model, which is threefold: (\textit{i}) \textbf{Reverser Engineering}, (\textit{ii}) \textbf{Reestructuring} and  (\textit{iii}) \textbf{Forward Engineering}. 

Firstly, all the artefacts of the legacy system must be transformed into PSMs by statically analyzing the legacy source code. Still in the first step, these PSMs are integrated into a KDM model, i.e., a PIM model, through a M2M transformation implemented using ATL. Secondly, the KnowDIME applies a set of model refactorings and model optimizations in this PIM. Afterwards, KnowDIME executes a set of M2M transformation taking as input the PIM and producing as output a model conforming to the KDM models into UML meta model. Finally, an improved system is obtained from this UML by means of a set of M2C transformation; additionally, the generated code can be complemented by the software engineer in accordance with the more detailed specifications of the application business rules, such as the implementation of specific behaviors and features not covered by the code generation.

We believe that KnowDIME makes a contribution to the challenges of Software Engineering which focuses on mechanisms to support the automation of software refactoring process. Long term future work involves conducting experiments to evaluate the level of maintenance provided by KnowDIME. It is worth highlighting that KnowDIME is open source and it can be downloaded at\textit{~www.dc.ufscar.br/$\sim$valter/crossfire}.


\section*{Acknowledgments}
	\input{tex/acknowledgment}


