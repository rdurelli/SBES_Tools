%It has been discusses how the engineer can use the ProLine-RM in order to assist the 


%CrossFIRE comprises two phases: (\textit{i}) DE and (\textit{ii}) AE. 
%The use of the ProLine-RM is twofold, DE and AE phases. 
In the DE phase, CrossFIRE provides an infrastructure in which all of the CFF artifacts is developed. Afterwards, the CrossFIRE allows the engineers share these artifacts. As a result, these artifacts will be stored in a remote repository to be reused in the AE phase. 

In the AE phase the CrossFIRE shows a list of all available CFFs that have been shared. Therefore, the engineer can pick out which CFF(s) can be reused in his base application. Next, the CrossFIRE provides a way to perform the download of the feature models related to each CFF. Through these feature models the engineer can pick out which features his base application really requires. As a consequence, the CrossFIRE downloads two artifacts, the necessaries chunks of code and a RM. Thus, the application engineer fills in the RM with the information needed by an member of a CFF's, and after that, it is possible to generate the final reuse code.

We believe that CrossFIRE increases the degree of reuse and allows the engineer to avoid dead code in his base application. Moreover, this infrastructure aims make the reuse activities easier. Long term future work involves conducting experiments to evaluate the level of reuse provided by CrossFIRE. It is worth highlighting that CrossFIRE is open source and it can be downloaded at\textit{~www.dc.ufscar.br/$\sim$valter/crossfire}.


%      AE phase the engineer picks out which CFF can be reused in his base application by using the ProLine-RM.

%By using this plug-in engineers can share CFFs. Furthermore, it allows the engineers pick out these CFFs and reuse them in a base application. This reuse is  these CFFs through a feature model that correspond the features that the application really requires through a feature model. After that, the plug-in provides a way to download only the necessaries pieces of code that the application requires, avoiding death code.


%The ProLine-RM has two mainly functionality. The first one is provides a way to share all artifacts of the CFF. and reuse of a full cycle of reuse of the CFF.

%The use of the ProLine-RM is twofold, DE and AE phases. In the DE phase it provides an environment in which all of the CFF artifacts is developed. Afterwards, by using the ProLine-RM these artifacts will be stored in a remote repository in order to be reused in the AE phase. 


%The ProLine-RM allows the engineer shares, manages of the CFF. and provides full cycle of reuse of the CFF. Furthermore, it is possible  it allows the engineer shares, manages  


%provides full cycle of reuse of the CFF. Moreover, it is impor- tant a tool that supplies a graphical way to allow the engineer examines previously if the features available in one CFF fulfill the application requirements.