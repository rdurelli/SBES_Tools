Companies have a vast number of operational legacy systems and these systems are not immune to software aging problems. However, they can not be discarded because they incorporate a lot of embodied knowledge due to years of maintenance and this constitutes a significant corporate asset. Furthermore, these kind of systems mainly consist of two artifacts: source code and databases. Usually, the maintenance of those artifacts is carried out through restructuring processes in isolated manners~\cite{Moser:2006}. Nevertheless, we claim that for a more effective maintenance of the whole system both should be analyzed and evolved jointly. 

Refactoring can be used in oder to deal with the maintenance of these systems. Martin Fowler~\cite{refactImpro} defines refactoring as ``A change made to the internal structure of software to make it easier to understand and cheaper to modify without changing its observable behavior''. In the literature there is a lot of studies that make refactoring at the level of source code. Consequently, nowadays almost all Integrated Development Environmentd (IDE) provide some kind of automated support for program refactoring. Empirical studies of refactoring have shown that it can improve maintainability~\cite{1510132} and reusability~\cite{Moser:2006} of legacy system. Not only does existing work suggest that refactoring is useful, but it also suggests that refactoring is a frequent practice~\cite{Murphy:2011}. Cherubini and colleagues` survey indicates that developers rate the importance of refactoring as equal to or greater than that of understanding code and producing documentation~\cite{Cherubini:2007}.

Recent researches have been shifted from the typical refactoring process to the so-called Architecture-Driven Modernization (ADM)~\cite{Ulrich:2010}. ADM has been proposed by OMG (Object Management Group). with the concept of modernizing legacy systems with a focus on all aspects of the current system architecture and the ability to transform current architectures to target architectures. In addition, ADM advocates conducting reengineering processes following the principles of Model-Driven Development (MDD)~\cite{Ulrich:2010:IST:1841736}, i.e., it treats all the software artifacts involved in the legacy system as models and can establish transformations among them.  Firstly a reverse engineering is performed starting from the source code and a model instance (\textbf{P}lataform \textbf{S}pecific \textbf{M}odel - PSM) is created. Next successive refinements (transformations) are applied to this model up to reach a good abstraction level (PSM) in model called KDM (\textbf{K}nowledge \textbf{D}iscovery \textbf{M}etamodel). Upon this model, several refactorings, optimizations and modifications can be performed in order to solve problems found in the legacy system. Secondly a forward engineering is carried out and the source code of the modernized target system is generated again. According to the OMG the most important artifact provided by ADM is the KDM metamodel, which is a multipurpose standard metamodel that represents all aspects of the existing IT (Information  Technology) architectures. The idea behind the standard KDM is that the community starts to create parsers from different languages to KDM. As a result everything that takes KDM as input can be considered platform and language-independent.

Accordingly, one of the challenges of Software Engineering focuses on specifications, architectures and mechanisms to support the automation of software reengineering process, aiming to reduce time and effort spent in this process. Furthermore, legacy systems, often with high maintenance costs due to scarcity or absence of documentation, may be restructured and ported to ADM, providing greater quality in development and maintenance of these systems. Motivated by these ideas, we put forward an infrastructure based on the metamodel KDM which support legacy system reengineering from databases queries and source code, with the objectives of reducing the time and effort in this process and providing migration of these systems to ADM. More specifically, this infrastructure analyses Structured Query Language (SQL) queries embedded in a legacy source code in order to restructure and re-organize the system by using design patterns, such as Data Access Object (DAO) and Service-Oriented Architecture (SOA).  This paper is organized as followed: Section 2 provides information related to the infrastructure - Section 3 the architecture of the infrastructure is depicted - in Section 4 there are related works and in Section 5 we conclude the paper with some remarks and future directions.


%Furthermore, the modernization of these artifacts  have a great impact in technological and economic terms (Sneed, 2005). In this context, maintenance based on evolutionary reengineering processes has been carried out (Bianchi et al., 2003). Moreover, the typical reengineering process has been shifted to the so-called Architecture-Driven Modernization(ADM)~\cite{Ulrich:2010} in the last years. ADM has been proposed by OMG (Object Management Group). ADM is the concept of modernizing legacy systems with a focus on all aspects of the current system architecture and the ability to transform current architectures to target architectures. ADM advocates conducting reengineering processes following the principles of Model-Driven Development (MDD)~\cite{Ulrich:2010:IST:1841736}, i.e., it treats all the software artifacts involved in the legacy system as models and can establish transformations among them. 





%Therefore, these system must be modernized in order to add new user requirements, new technological or even platform migration.

%Software systems are considered legacy when their maintenance costs are raised to undesirable levels but they are still valuable for organizations. However, they can not be discarded because they incorporate a lot of embodied knowledge due to years of maintenance and this constitutes a significant corporate asset. As these systems still provide significant business value, they must then be modernized/re-engineered so that their maintenance costs can be manageable and they can keep on assisting in the regular daily activities. 

%The first task that must be performed in order to carrying out a software modernization is understand the legacy system. It is not a trivial task, in fact studies estimate that between $50$ percent and $90$ percent of software maintenance involves developing an understanding of the software being maintained~\cite{Tilley95perspectiveson}, thus several approaches have been developed to support software engineers in the comprehension of systems where reverse engineering (RE) is one of them~\cite{Canfora2011}. RE supports program comprehension by using techniques that explore the source code to find relevant information related to functional and non-functional features~\cite{chikofskyTax}.

%In this context, OMG (Object Management Group) has employed a lot of effort to define standards in the modernization process, creating the concept of ADM (Architecture-Driven Modernization). ADM follows the MDD (Model-Driven Development)~\cite{Ulrich:2010:IST:1841736}~\cite{5440163} guidelines and comprises two major steps. Firstly a reverse engineering is performed starting from the source code and a model instance (PSM) is created. Next successive refinements (transformations) are applied to this model up to reach a good abstraction level (PSM or CIM) in model called KDM (Knowledge Discovery Metamodel). Upon this model, several refactorings, optimizations and modifications can be performed in order to solve problems found in the legacy system. Secondly a forward engineering is carried out and the source code of the modernized target system is generated again. According to the OMG the most important artifact provided by ADM is the KDM metamodel, which is a multipurpose standard metamodel that represents all aspects of the existing IT (Information  Technology) architectures. The idea behind the standard KDM is that the community starts to create parsers from different languages to KDM. As a result everything that takes KDM as input can be considered platform and language-independent. For example, a refactoring catalogue for KDM can be used for refactoring systems implemented in different languages. 


