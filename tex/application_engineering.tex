%The AE phase is represented by the numbers 1 to 3 as shown in Figure~\ref{fig:proline}. 
%As stated previously, in the AE is wherein the reuse is started effectively.

Firstly, the engineer has to look in the repository and determine whether there are some CFF(s), which can be reused, to make easier and faster the development of the base application. 
In order to assist this activity, ProLine-RM provides a table, which depicts all CFFs that have been uploaded by the engineers in the DE phase.
Figure~\ref{fig:proline}(B) shows this table. 
As can be seen there are six different CFFs - persistence, security, distribution, concurrency, logging and 	Business, respectively. 
In addition, ProLine-RM also shows descriptions for each of the selected CFFs by clicking on the button ``Description''. 
Using this description the engineer can choose the CFFs. 
Nevertheless, if this description is not enough to help the engineer takes a decision on reusing the CFF, ProLine-RM supplies a way visualizes the feature model related to selected CFF by clicking on the button ``View''. 
As is shown in Figure~\ref{fig:proline}(B) the ``persistence'' CFF is highlighted meaning that it has been chosen. 
Next, the ``Download'' button has to be clicked to transfer the feature model belonging to the CFF chosen from the remote repository to the engineer computer.

Secondly, to reuse the CFF its features must be chosen by the engineer aiming to specify explicitly which features will be used in the application base. This is important because usually the CFFs have a great deal of features that probably will not be used in the application base. To assist this activity ProLine-RM uses a file named ``configuration file''. This file is created based on the feature model downloaded. ProLine-RM reads the feature model and creates a ``tree hierarchy''. By using this ``configuration file'' features can be chosen by the engineer. The ``configuration file'' related to the ``persistence'' CFF is shown in Figure~\ref{fig:proline}(D). Moreover, it is interesting to provide a way to validate if the selected features match a valid combination for the instantiation of a member of the CFF, since, certain combinations of features may not lead to useful variants (e.g., in our example the ``persistence'' CFF only a single database connection may be used). ProLine-RM support this validation automatically. As shown in Figure~\ref{fig:proline}(D) once the engineer has chosen the features (represented by ``+''), the resulting variant and constraints are generated automatically (represented by ``-''). 

Finally, the engineer has to submit this validated ``configuration file'' to the remote repository where all CFFs persist. Using this ``configuration file'' the repository will carry out an algorithm. This algorithm aims to extract two artifacts, the codes (e.g., classes, aspects, and packages) related to the features that was chosen by the engineer and a `Reuse Model' (RM), which it is useful to assist the instantiation of CFF's member. After that, these artifacts are sent by the repository to the ProLine-RM. %As can be seen in Figure~\ref{fig:proline}(3) the repository sent only the packages related to the features selected and specified through the ``configuration file'' i.e., the features Persistence, Connection and MySQL.
The application engineer has to filled in the RM. For instance, the value ``base.Custormer.initial'' is a method of the base application and was inserted by the application engineer in the third line of box ``Connection Opening''. After the application engineer fills in the RM with the information needed by an member of a CFF, it is possible to generate the final reuse code.

     
%To the best of our knowledge, there is no an approach that shares, managements, provides full cycle of reuse of the CFF and even supplies a way to examine previously if the features available in one CFF fulfill the application requirements. In order to overcome these absence we put forward an approach and a tool named Proline-RM acronym for Product Line-Repository Manager, which aims to increase the level of manager and accelerate the instantiation of members belonging to a given CFF. The use of the approach is twofold, the Domain Engineering (DE) where all artifacts are developed and upload to a remote server, and the Application Engineering (AE), where the reuse is done effectively. 