 The approach proposed by Cechticky \textit{et al.}  \citep{Cechticky:2003:GAF:954186.954203} allows object-oriented application framework reuse by using a tool called OBS Instantiation Environment. That tool supports graphical models do define the settings of the expected application to be generated. The model to code transformation generates a new application  that reuses the framework. The proposal found in this paper differs from their approach on the following topics: (\textit{i}) their approach is restricted to frameworks known during the development of the tool; (\textit{ii}) it does not use aspect orientation; (\textit{iii}) the reuse process is applied on application frameworks, %, therefore, %a completely new application is generated.
which are used to create new applications.


 Another approach was proposed by Oliveira \textit{et al.}  \citep{Oliveira:2011:RET:2039458.2039832}. Their approach can be applied to a greater number of object oriented frameworks. After the framework development, the framework developer may use the approach to ease the reuse by writing the cookbook in a formal language known as Reuse Definition Language (RDL) which also can be used to generate the source code.
This process allows to select the variabilities and resources during reuse, as long as the framework engineer specifies the RDL code correctly.