
%In this section is shown the architecture of the ProLine-RM as well as an description of its components. 
In Figure~\ref{fig:architecture} is depicted the architecture of our infrastructure. As shown in this figure, we devised it on the top of the Eclipse Platform and used both Java and Groovy as programming language. Moreover, we used Eclipse Modeling Framework (EMF)\footnote{http://www.eclipse.org/modeling/emf/} to create the SQL model and to reutilize the UML model. MoDisco is used by infrastructure since it provides an
\textbf{A}pplication \textbf{P}rogramming \textbf{I}nterface - (API) to easily access the KDM model. 

\begin{figure}[!h]
\centering
  % Requires \usepackage{graphicx}
 \includegraphics[scale=0.8]{Figuras/Arquitetura_da_Ferramenta}
\caption{Architecture}
\label{fig:architecture}
\end{figure}



\textbf{A}Nother \textbf{T}ool for \textbf{L}anguage \textbf{R}ecognition -  (ANTLR) is used herein to create parsers to obtain information related to the legacy system's artifacts. Therefore, two parser were developed: (\textit{i}) the first takes as input a Java grammar and generates as output an AST and (\textit{ii}) the second parser is a extension of the first one to identify SQL embedded in the legacy system's source code, i.e., it takes as input a Java source-code and generates as output an AST which contains informations such as, tables, columns, primary keys, etc. Then, to transform these AST in a PSM we used an API provided by EMF.
 

  %tha  both the source-code of the legacy system and database/SQL queries embedded in into PSM models.


%As can be seen in Figure~\ref{fig:architecture}, all artifacts (source code, feature model, RM and RRM) that the CrossFIRE provides are persisted in a database. These artifacts are persisted in a remote server, available to be reused in the AE phase. This remote server is a physical computer, which is dedicated to run the RESTful API. Therefore, to send these artifacts by the server we have used this API as web service to cache the representation of all artifacts. This server receives requests of the CrossFIRE through RESTful, processes database queries and sends a response to the CrossFIRE by using RESTful as well. Furthermore, we have used Java Persistence API (JPA) 2.0 to deal with the way relational data is mapped to Java objects. To implement the database of the server, the MySQL was chosen.

