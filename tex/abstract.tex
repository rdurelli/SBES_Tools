%Company runs systems that have been implemented a long time ago. These systems usually are still under adaptation and maintenance to address current needs. Very often, adapting legacy software systems to new requirements needs to make use of new technological advances. 
Legacy systems mainly consist of two kinds of artifacts: source code and databases. Usually, the maintenance of those artifacts is carried out through refactoring processes in isolated manners and without following any kind of standardization. In order to provide a more effective maintenance of the whole system both artifacts should be analyzed, evolved jointly and standardized way. As is known refactoring is a very useful process for dealing with software aging problem since it can improve maintainability and reusability of these system. Recently, researches have been shifted from the typical refactoring processes to the so-called Architecture-Driven Modernization (ADM), which is an approach that follows the all the principles of Model-Driven Development (MDD). As one of the challenges of Software Engineering focuses on mechanisms to support the automation of software refactoring process in this paper we put forward KnowDIME to assist the modernization of a legacy system based on ADM, which uses the Knowledge Discovery Metamodel (KDM) standard. This infrastructure analyses Structured Query Language (SQL) queries embedded in a legacy source code in order to restructure and re-organize the system by using design patterns, such as Data Access Object (DAO) and Service-Oriented Architecture (SOA). 