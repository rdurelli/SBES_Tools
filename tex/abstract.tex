%Company runs systems that have been implemented a long time ago. These systems usually are still under adaptation and maintenance to address current needs. Very often, adapting legacy software systems to new requirements needs to make use of new technological advances. 
Legacy systems mainly consist of two kinds of artifacts: source code and databases. Usually, the maintenance of those artifacts is carried out through refactoring processes in isolated manners. Nevertheless, for a more effective maintenance of the whole system both should be analyzed and evolved jointly. Recent researches have been shifted from the typical refactoring processes to the so-called Architecture-Driven Modernization (ADM), which is an approach that follows the principles of Model-Driven Development (MDD). In addition, one of the challenges of Software Engineering focuses on mechanisms to support the automation of software refactoring process. Thus, in this paper we put forward an infrastructure  to assist the modernization of a legacy system by using ADM.%both source code and database legacy systems. 
This infrastructure analyses Structured Query Language (SQL) queries embedded in a legacy source code in order to restructure and re-organize the system by using design patterns, such as Data Access Object (DAO) and Service-Oriented Architecture (SOA). The infrastructure is an Eclipse plug-in and can be extended with new types of refactoring.