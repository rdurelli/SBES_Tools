
%% bare_conf.tex
%% V1.3
%% 2007/01/11
%% by Michael Shell
%% See:
%% http://www.michaelshell.org/
%% for current contact information.
%%
%% This is a skeleton file demonstrating the use of IEEEtran.cls
%% (requires IEEEtran.cls version 1.7 or later) with an IEEE conference paper.
%%
%% Support sites:
%% http://www.michaelshell.org/tex/ieeetran/
%% http://www.ctan.org/tex-archive/macros/latex/contrib/IEEEtran/
%% and
%% http://www.ieee.org/

%%*************************************************************************
%% Legal Notice:
%% This code is offered as-is without any warranty either expressed or
%% implied; without even the implied warranty of MERCHANTABILITY or
%% FITNESS FOR A PARTICULAR PURPOSE! 
%% User assumes all risk.
%% In no event shall IEEE or any contributor to this code be liable for
%% any damages or losses, including, but not limited to, incidental,
%% consequential, or any other damages, resulting from the use or misuse
%% of any information contained here.
%%
%% All comments are the opinions of their respective authors and are not
%% necessarily endorsed by the IEEE.
%%
%% This work is distributed under the LaTeX Project Public License (LPPL)
%% ( http://www.latex-project.org/ ) version 1.3, and may be freely used,
%% distributed and modified. A copy of the LPPL, version 1.3, is included
%% in the base LaTeX documentation of all distributions of LaTeX released
%% 2003/12/01 or later.
%% Retain all contribution notices and credits.
%% ** Modified files should be clearly indicated as such, including  **
%% ** renaming them and changing author support contact information. **
%%
%% File list of work: IEEEtran.cls, IEEEtran_HOWTO.pdf, bare_adv.tex,
%%                    bare_conf.tex, bare_jrnl.tex, bare_jrnl_compsoc.tex
%%*************************************************************************

% *** Authors should verify (and, if needed, correct) their LaTeX system  ***
% *** with the testflow diagnostic prior to trusting their LaTeX platform ***
% *** with production work. IEEE's font choices can trigger bugs that do  ***
% *** not appear when using other class files.                            ***
% The testflow support page is at:
% http://www.michaelshell.org/tex/testflow/



% Note that the a4paper option is mainly intended so that authors in
% countries using A4 can easily print to A4 and see how their papers will
% look in print - the typesetting of the document will not typically be
% affected with changes in paper size (but the bottom and side margins will).
% Use the testflow package mentioned above to verify correct handling of
% both paper sizes by the user's LaTeX system.
%
% Also note that the "draftcls" or "draftclsnofoot", not "draft", option
% should be used if it is desired that the figures are to be displayed in
% draft mode.
%
\documentclass[conference]{IEEEtran}
% Add the compsoc option for Computer Society conferences.
%
% If IEEEtran.cls has not been installed into the LaTeX system files,
% manually specify the path to it like:
% \documentclass[conference]{../sty/IEEEtran}


 

% Some very useful LaTeX packages include:
% (uncomment the ones you want to load)


% *** MISC UTILITY PACKAGES ***
%
%\usepackage{ifpdf}
% Heiko Oberdiek's ifpdf.sty is very useful if you need conditional
% compilation based on whether the output is pdf or dvi.
% usage:
% \ifpdf
%   % pdf code
% \else
%   % dvi code
% \fi
% The latest version of ifpdf.sty can be obtained from:
% http://www.ctan.org/tex-archive/macros/latex/contrib/oberdiek/
% Also, note that IEEEtran.cls V1.7 and later provides a builtin
% \ifCLASSINFOpdf conditional that works the same way.
% When switching from latex to pdflatex and vice-versa, the compiler may
% have to be run twice to clear warning/error messages.






% *** CITATION PACKAGES ***
%
\usepackage{cite}
% cite.sty was written by Donald Arseneau
% V1.6 and later of IEEEtran pre-defines the format of the cite.sty package
% \cite{} output to follow that of IEEE. Loading the cite package will
% result in citation numbers being automatically sorted and properly
% "compressed/ranged". e.g., [1], [9], [2], [7], [5], [6] without using
% cite.sty will become [1], [2], [5]--[7], [9] using cite.sty. cite.sty's
% \cite will automatically add leading space, if needed. Use cite.sty's
% noadjust option (cite.sty V3.8 and later) if you want to turn this off.
% cite.sty is already installed on most LaTeX systems. Be sure and use
% version 4.0 (2003-05-27) and later if using hyperref.sty. cite.sty does
% not currently provide for hyperlinked citations.
% The latest version can be obtained at:
% http://www.ctan.org/tex-archive/macros/latex/contrib/cite/
% The documentation is contained in the cite.sty file itself.






% *** GRAPHICS RELATED PACKAGES ***
%
\ifCLASSINFOpdf
   \usepackage[pdftex]{graphicx}
  % declare the path(s) where your graphic files are
  % \graphicspath{{../pdf/}{../jpeg/}}
  % and their extensions so you won't have to specify these with
  % every instance of \includegraphics
  % \DeclareGraphicsExtensions{.pdf,.jpeg,.png}
\else
  % or other class option (dvipsone, dvipdf, if not using dvips). graphicx
  % will default to the driver specified in the system graphics.cfg if no
  % driver is specified.
   \usepackage[dvips]{graphicx}
  % declare the path(s) where your graphic files are
  % \graphicspath{{../eps/}}
  % and their extensions so you won't have to specify these with
  % every instance of \includegraphics
  % \DeclareGraphicsExtensions{.eps}
\fi
% graphicx was written by David Carlisle and Sebastian Rahtz. It is
% required if you want graphics, photos, etc. graphicx.sty is already
% installed on most LaTeX systems. The latest version and documentation can
% be obtained at: 
% http://www.ctan.org/tex-archive/macros/latex/required/graphics/
% Another good source of documentation is "Using Imported Graphics in
% LaTeX2e" by Keith Reckdahl which can be found as epslatex.ps or
% epslatex.pdf at: http://www.ctan.org/tex-archive/info/
%
% latex, and pdflatex in dvi mode, support graphics in encapsulated
% postscript (.eps) format. pdflatex in pdf mode supports graphics
% in .pdf, .jpeg, .png and .mps (metapost) formats. Users should ensure
% that all non-photo figures use a vector format (.eps, .pdf, .mps) and
% not a bitmapped formats (.jpeg, .png). IEEE frowns on bitmapped formats
% which can result in "jaggedy"/blurry rendering of lines and letters as
% well as large increases in file sizes.
%
% You can find documentation about the pdfTeX application at:
% http://www.tug.org/applications/pdftex





% *** MATH PACKAGES ***
%
%\usepackage[cmex10]{amsmath}
% A popular package from the American Mathematical Society that provides
% many useful and powerful commands for dealing with mathematics. If using
% it, be sure to load this package with the cmex10 option to ensure that
% only type 1 fonts will utilized at all point sizes. Without this option,
% it is possible that some math symbols, particularly those within
% footnotes, will be rendered in bitmap form which will result in a
% document that can not be IEEE Xplore compliant!
%
% Also, note that the amsmath package sets \interdisplaylinepenalty to 10000
% thus preventing page breaks from occurring within multiline equations. Use:
%\interdisplaylinepenalty=2500
% after loading amsmath to restore such page breaks as IEEEtran.cls normally
% does. amsmath.sty is already installed on most LaTeX systems. The latest
% version and documentation can be obtained at:
% http://www.ctan.org/tex-archive/macros/latex/required/amslatex/math/





% *** SPECIALIZED LIST PACKAGES ***
%
%\usepackage{algorithmic}
% algorithmic.sty was written by Peter Williams and Rogerio Brito.
% This package provides an algorithmic environment fo describing algorithms.
% You can use the algorithmic environment in-text or within a figure
% environment to provide for a floating algorithm. Do NOT use the algorithm
% floating environment provided by algorithm.sty (by the same authors) or
% algorithm2e.sty (by Christophe Fiorio) as IEEE does not use dedicated
% algorithm float types and packages that provide these will not provide
% correct IEEE style captions. The latest version and documentation of
% algorithmic.sty can be obtained at:
% http://www.ctan.org/tex-archive/macros/latex/contrib/algorithms/
% There is also a support site at:
% http://algorithms.berlios.de/index.html
% Also of interest may be the (relatively newer and more customizable)
% algorithmicx.sty package by Szasz Janos:
% http://www.ctan.org/tex-archive/macros/latex/contrib/algorithmicx/




% *** ALIGNMENT PACKAGES ***
%
\usepackage{array}
% Frank Mittelbach's and David Carlisle's array.sty patches and improves
% the standard LaTeX2e array and tabular environments to provide better
% appearance and additional user controls. As the default LaTeX2e table
% generation code is lacking to the point of almost being broken with
% respect to the quality of the end results, all users are strongly
% advised to use an enhanced (at the very least that provided by array.sty)
% set of table tools. array.sty is already installed on most systems. The
% latest version and documentation can be obtained at:
% http://www.ctan.org/tex-archive/macros/latex/required/tools/


%\usepackage{mdwmath}
%\usepackage{mdwtab}
% Also highly recommended is Mark Wooding's extremely powerful MDW tools,
% especially mdwmath.sty and mdwtab.sty which are used to format equations
% and tables, respectively. The MDWtools set is already installed on most
% LaTeX systems. The lastest version and documentation is available at:
% http://www.ctan.org/tex-archive/macros/latex/contrib/mdwtools/


% IEEEtran contains the IEEEeqnarray family of commands that can be used to
% generate multiline equations as well as matrices, tables, etc., of high
% quality.


%\usepackage{eqparbox}
% Also of notable interest is Scott Pakin's eqparbox package for creating
% (automatically sized) equal width boxes - aka "natural width parboxes".
% Available at:
% http://www.ctan.org/tex-archive/macros/latex/contrib/eqparbox/





% *** SUBFIGURE PACKAGES ***
%\usepackage[tight,footnotesize]{subfigure}
% subfigure.sty was written by Steven Douglas Cochran. This package makes it
% easy to put subfigures in your figures. e.g., "Figure 1a and 1b". For IEEE
% work, it is a good idea to load it with the tight package option to reduce
% the amount of white space around the subfigures. subfigure.sty is already
% installed on most LaTeX systems. The latest version and documentation can
% be obtained at:
% http://www.ctan.org/tex-archive/obsolete/macros/latex/contrib/subfigure/
% subfigure.sty has been superceeded by subfig.sty.



%\usepackage[caption=false]{caption}
%\usepackage[font=footnotesize]{subfig}
% subfig.sty, also written by Steven Douglas Cochran, is the modern
% replacement for subfigure.sty. However, subfig.sty requires and
% automatically loads Axel Sommerfeldt's caption.sty which will override
% IEEEtran.cls handling of captions and this will result in nonIEEE style
% figure/table captions. To prevent this problem, be sure and preload
% caption.sty with its "caption=false" package option. This is will preserve
% IEEEtran.cls handing of captions. Version 1.3 (2005/06/28) and later 
% (recommended due to many improvements over 1.2) of subfig.sty supports
% the caption=false option directly:
%\usepackage[caption=false,font=footnotesize]{subfig}
%
% The latest version and documentation can be obtained at:
% http://www.ctan.org/tex-archive/macros/latex/contrib/subfig/
% The latest version and documentation of caption.sty can be obtained at:
% http://www.ctan.org/tex-archive/macros/latex/contrib/caption/
\usepackage{natbib}



% *** FLOAT PACKAGES ***
%
%\usepackage{fixltx2e}
% fixltx2e, the successor to the earlier fix2col.sty, was written by
% Frank Mittelbach and David Carlisle. This package corrects a few problems
% in the LaTeX2e kernel, the most notable of which is that in current
% LaTeX2e releases, the ordering of single and double column floats is not
% guaranteed to be preserved. Thus, an unpatched LaTeX2e can allow a
% single column figure to be placed prior to an earlier double column
% figure. The latest version and documentation can be found at:
% http://www.ctan.org/tex-archive/macros/latex/base/



%\usepackage{stfloats}
% stfloats.sty was written by Sigitas Tolusis. This package gives LaTeX2e
% the ability to do double column floats at the bottom of the page as well
% as the top. (e.g., "\begin{figure*}[!b]" is not normally possible in
% LaTeX2e). It also provides a command:
%\fnbelowfloat
% to enable the placement of footnotes below bottom floats (the standard
% LaTeX2e kernel puts them above bottom floats). This is an invasive package
% which rewrites many portions of the LaTeX2e float routines. It may not work
% with other packages that modify the LaTeX2e float routines. The latest
% version and documentation can be obtained at:
% http://www.ctan.org/tex-archive/macros/latex/contrib/sttools/
% Documentation is contained in the stfloats.sty comments as well as in the
% presfull.pdf file. Do not use the stfloats baselinefloat ability as IEEE
% does not allow \baselineskip to stretch. Authors submitting work to the
% IEEE should note that IEEE rarely uses double column equations and
% that authors should try to avoid such use. Do not be tempted to use the
% cuted.sty or midfloat.sty packages (also by Sigitas Tolusis) as IEEE does
% not format its papers in such ways.




% *** PDF, URL AND HYPERLINK PACKAGES ***
%
%\usepackage{url}
% url.sty was written by Donald Arseneau. It provides better support for
% handling and breaking URLs. url.sty is already installed on most LaTeX
% systems. The latest version can be obtained at:
% http://www.ctan.org/tex-archive/macros/latex/contrib/misc/
% Read the url.sty source comments for usage information. Basically,
% \url{my_url_here}.


\usepackage[latin9]{inputenc} 


% *** Do not adjust lengths that control margins, column widths, etc. ***
% *** Do not use packages that alter fonts (such as pslatex).         ***
% There should be no need to do such things with IEEEtran.cls V1.6 and later.
% (Unless specifically asked to do so by the journal or conference you plan
% to submit to, of course. )
%\usepackage{algorithmic}
\usepackage{listings}
%tabelas com tamanho expans�vel
\usepackage{tabularx}
%multicoluna/linha
\usepackage{multirow}
\usepackage[table]{xcolor}
%\usepackage{rotating}
% correct bad hyphenation here
\hyphenation{op-tical net-works semi-conduc-tor}

%funcoes, constantes e outras definicoes
\newcommand{\papertitle}{KnowDIME: An Infrastructure based on Architecture-Driven Modernization for Improving Legacy System}

\begin{document}
%
% paper title
% can use linebreaks \\ within to get better formatting as desired
%\title{Model Driven Approach for Crosscutting Framework Reuse}
%
\title{\papertitle}

% author names and affiliations
% use a multiple column layout for up to three different
% affiliations
%\author{\ \IEEEauthorblockN{Thiago Gottardi and Rafael Serapilha Durelli }
%\IEEEauthorblockA{
%%Departamento de Computa\c{c}\~{a}o -- Universidade Federal de S\~{a}o Carlos
%Computing Department -- Federal University of  S\~{a}o Carlos
%  (UFSCar)\\ Caixa Postal 676 -- 13.565-905 -- S\~{a}o Carlos -- SP -- Brazil} 
%
%  \IEEEauthorblockB{
%  Trabalhador S�o-carlense Av., 400, S�o Carlos-SP
%  }
%  
%  }

%\author{\IEEEauthorblockN{Thiago Gottardi}
%\IEEEauthorblockA{Departamento de Computa\c{c}\~{a}o\\
%Universidade Federal de  S\~{a}o Carlos\\
%Caixa Postal 676 -- 13.565-905 \\ S\~{a}o Carlos -- SP -- Brazil\\
%Email: thiago\_gottardi@dc.ufscar.br}
%\and
%\IEEEauthorblockN{Rafael Serapilha Durelli}
%\IEEEauthorblockA{Instituto de  Ci\^{e}ncias Matem\'{a}ticas e  Computa\c{c}\~{a}o \\
%Universidade de S\~{a}o Paulo\\
%Av. Trabalhador S\~{a}o Carlense, 400\\ S\~{a}o Carlos -- SP -- Brazil\\
%Email: rsdurelli@icmc.usp.br}
%\and
%\IEEEauthorblockN{Oscar L\'{o}pez Pastor}
%\IEEEauthorblockA{n\\
%Universidad Politecnica de Valencia\\
%Camino de Vera s/n, Valencia, Spain\\
%Email: opastor@dsic.upv.es}
%\and
%\IEEEauthorblockN{Valter Vieira de Camargo}
%\IEEEauthorblockA{Departamento de Computa\c{c}\~{a}o\\
%Universidade Federal de  S\~{a}o Carlos\\
%Caixa Postal 676 -- 13.565-905 \\ S\~{a}o Carlos -- SP -- Brazil\\
%Email: valter@dc.ufscar.br}}

% conference papers do not typically use \thanks and this command
% is locked out in conference mode. If really needed, such as for
% the acknowledgment of grants, issue a \IEEEoverridecommandlockouts
% after \documentclass

% for over three affiliations, or if they all won't fit within the width
% of the page, use this alternative format:
% 32 323232220001111111123
%\author{\IEEEauthorblockN{Michael Shell\IEEEauthorrefmark{1},
%Homer Simpson\IEEEauthorrefmark{2},
%James Kirk\IEEEauthorrefmark{3}, 
%Montgomery Scott\IEEEauthorrefmark{3} and
%Eldon Tyrell\IEEEauthorrefmark{4}}
%\IEEEauthorblockA{\IEEEauthorrefmark{1}School of Electrical and Computer Engineering\\
%Georgia Institute of Technology,
%Atlanta, Georgia 30332--0250\\ Email: see http://www.michaelshell.org/contact.html}
%\IEEEauthorblockA{\IEEEauthorrefmark{2}Twentieth Century Fox, Springfield, USA\\
%Email: homer@thesimpsons.com}
%\IEEEauthorblockA{\IEEEauthorrefmark{3}Starfleet Academy, San Francisco, California 96678-2391\\
%Telephone: (800) 555--1212, Fax: (888) 555--1212}
%\IEEEauthorblockA{\IEEEauthorrefmark{4}Tyrell Inc., 123 Replicant Street, Los Angeles, California 90210--4321}}

\author{\IEEEauthorblockN{Thiago Gottardi\IEEEauthorrefmark{1},
Rafael Serapilha Durelli\IEEEauthorrefmark{2},
Oscar Pastor L\'{o}pez\IEEEauthorrefmark{3} and
Valter Vieira de Camargo\IEEEauthorrefmark{1}}
\IEEEauthorblockA{\IEEEauthorrefmark{1}Departamento de Computa\c{c}\~{a}o,
Universidade Federal de  S\~{a}o Carlos,\\
Caixa Postal 676 -- 13.565-905, S\~{a}o Carlos -- SP -- Brazil\\ Email: \{thiago\_gottardi,valter\}@dc.ufscar.br}
\IEEEauthorblockA{\IEEEauthorrefmark{2}Instituto de  Ci\^{e}ncias Matem\'{a}ticas e  Computa\c{c}\~{a}o,
Universidade de S\~{a}o Paulo,\\
Av. Trabalhador S\~{a}o Carlense, 400, S\~{a}o Carlos -- SP -- Brazil\\
Email: rsdurelli@icmc.usp.br}
\IEEEauthorblockA{\IEEEauthorrefmark{3}Universidad Politecnica de Valencia,
Camino de Vera s/n, Valencia, Spain\\
Email: opastor@dsic.upv.es}}




% use for special paper notices
%\IEEEspecialpapernotice{(Invited Paper)}




% make the title area
\maketitle

 




\begin{abstract}
%Company runs systems that have been implemented a long time ago. These systems usually are still under adaptation and maintenance to address current needs. Very often, adapting legacy software systems to new requirements needs to make use of new technological advances. 
Legacy systems mainly consist of two kinds of artifacts: source code and databases. Usually, the maintenance of those artifacts is carried out through refactoring processes in isolated manners and without following any kind of standardization. In order to provide a more effective maintenance of the whole system both artifacts should be analyzed, evolved jointly and standardized way. As is known refactoring is a very useful process for dealing with software aging problem since it can improve maintainability and reusability of these system. Recently, researches have been shifted from the typical refactoring processes to the so-called Architecture-Driven Modernization (ADM), which is an approach that follows the all the principles of Model-Driven Development (MDD). As one of the challenges of Software Engineering focuses on mechanisms to support the automation of software refactoring process in this paper we put forward KnowDIME to assist the modernization of a legacy system based on ADM, which uses the Knowledge Discovery Metamodel (KDM) standard. This infrastructure analyses Structured Query Language (SQL) queries embedded in a legacy source code in order to restructure and re-organize the system by using design patterns, such as Data Access Object (DAO) and Service-Oriented Architecture (SOA). 
\end{abstract}
% IEEEtran.cls defaults to using nonbold math in the Abstract.
% This preserves the distinction between vectors and scalars. However,
% if the conference you are submitting to favors bold math in the abstract,
% then you can use LaTeX's standard command \boldmath at the very start
% of the abstract to achieve this. Many IEEE journals/conferences frown on
% math in the abstract anyway.

% no keywords




% For peer review papers, you can put extra information on the cover
% page as needed:
% \ifCLASSOPTIONpeerreview
% \begin{center} \bfseries EDICS Category: 3-BBND \end{center}
% \fi
%
% For peerreview papers, this IEEEtran command inserts a page break and
% creates the second title. It will be ignored for other modes.
\IEEEpeerreviewmaketitle

\section{Introduction}
	Companies have a vast number of operational legacy systems and these systems are not immune to software aging problems. However, they can not be discarded because they incorporate a lot of embodied knowledge due to years of maintenance and this constitutes a significant corporate asset. Furthermore, these kind of systems mainly consist of two artifacts: source code and databases. Usually, the maintenance of those artifacts is carried out through restructuring processes in isolated manners~\cite{Moser:2006}. Nevertheless, we claim that for a more effective maintenance of the whole system both should be analyzed and evolved jointly. 

Refactoring can be used in order to deal with the maintenance of these systems. Martin Fowler~\cite{refactImpro} defines refactoring as ``A change made to the internal structure of software to make it easier to understand and cheaper to modify without changing its observable behavior''. In the literature there is a lot of studies that make refactoring at the level of source code. Empirical studies of refactoring have shown that it can improve maintainability~\cite{1510132} and reusability~\cite{Moser:2006} of legacy system. Not only does existing work suggest that refactoring is useful, but it also suggests that refactoring is a frequent practice~\cite{Murphy:2011}. Cherubini and colleagues` survey indicates that developers rate the importance of refactoring as equal to or greater than that of understanding code and producing documentation~\cite{Cherubini:2007}.

In a parallel research line, researchers have been shifted from the typical refactoring process to the so-called Architecture-Driven Modernization (ADM)~\cite{Ulrich:2010}. ADM has been proposed by OMG (Object Management Group) with the concept of modernizing legacy systems with a focus on all aspects of the current system architecture and the ability to transform current architectures to target architectures. In addition, ADM advocates conducting reengineering processes following the principles of Model-Driven Development (MDD)~\cite{Ulrich:2010:IST:1841736}, i.e., it treats all the software artifacts involved in the legacy system as models and can establish transformations among them.  Firstly a reverse engineering is performed starting from the source code and a model instance (\textbf{P}lataform \textbf{S}pecific \textbf{M}odel - PSM) is created. Next successive refinements (transformations) are applied to this model up to reach a good abstraction level (PSM) in model called KDM (\textbf{K}nowledge \textbf{D}iscovery \textbf{M}etamodel). Upon this model, several refactorings, optimizations and modifications can be performed in order to solve problems found in the legacy system. Secondly a forward engineering is carried out and the source code of the modernized target system is generated again. According to the OMG the most important artifact provided by ADM is the KDM metamodel, which is a multipurpose standard metamodel that represents all aspects of the existing IT (Information  Technology) architectures. The idea behind the standard KDM is that the community starts to create parsers from different languages to KDM. As a result everything that takes KDM as input can be considered platform and language-independent.

Accordingly, one of the challenges of Software Engineering focuses on specifications, architectures and mechanisms to support the automation of software reengineering process, aiming to reduce time and effort spent in this process. Furthermore, legacy systems, often with high maintenance costs due to scarcity or absence of documentation, may be restructured and ported to ADM, providing greater quality in development and maintenance of these systems. Motivated by these ideas, we put forward an infrastructure based on the metamodel KDM which support legacy system reengineering from databases queries and source code, with the objectives of reducing the time and effort in this process and providing migration of these systems to ADM. More specifically, this infrastructure analyses Structured Query Language (SQL) queries embedded in a legacy source code in order to restructure and re-organize the system by using design patterns, such as Data Access Object (DAO) and Service-Oriented Architecture (SOA).  This paper is organized as followed: Section 2 provides information related to the infrastructure - Section 3 the architecture of the infrastructure is depicted - in Section 4 there are related works and in Section 5 we conclude the paper with some remarks and future directions.

	%This paper is structured as follows: in Section II, Crosscutting Frameworks are explained; in Section III, the Proposed Model and the Reuse Process are shown; in Section IV, a tool to support the process is used to reuse a persistence framework as an Example; in Section V, an empirical evaluation is presented; in Section VI, there are related works and in Section VII, there are the conclusions.
	
	
\section{Software Restructuring\label{software restructuring}}
	Perhaps the most common of all software engineering activities is the modifications of software. Unfortunately, software modification, i.e., software maintenance, often leaves behind software that is difficult to understand for those other than its author. In this context, software restructuring is a field that seeks to reverse these effects on software. 

More specifically, software restructuring is the modification of software to make the software easier to understand and to change, or less susceptible to error when future changes are made (ref). In other words, it is the process of re-organizing the logical structure of existing software system to improve specific attributes~\cite{kang1999}. Some examples of software restructuring are improving coding style, editing documentation, transforming program components (moving class, creating class, etc). The central idea of restructuring is the action of transformation. According to~\cite{Eloff2002} a transformation can be defined formally as a function that receives a program, P, as input and produces a new program, P'. Thus, P' is said to be functionally equivalent to P $\Leftrightarrow$ P' exhibits identical behavior  to P for all defined inputs of P. Finally, T is called a meaning preserving transformation if P' $\equiv$ P.

According to Griswold's experiments programmers tends to not only commit syntactic errors and behave inconsistently, they usually ignore the global impact of the changes they make 

Manually restructuring software may have undesirable, and often unforeseen results that can affect the be- haviour of a system. Griswold’s experiments found that programmers not only commit syntactic errors and behave inconsistently, but they also ignore the global impact of the changes they make [Griswold 1991]. Fur- thermore, manual techniques demand the maintainer to guarantee the preservation of the system’s behaviour.


\section{Model-Driven Restructuring\label{MDD}}
	According to Griswold's experiments programmers tends to not only commit syntactic errors an behave inconsistently, they usually ignore the global impact of the changes they make during the restructuring process. Moreover, he also argues that manual restructuring is an error-prone and expensive activity~\cite{grisswold}. Therefore, software engineers have applied Model-Driven Development (MDD) technologies to software restructuring to deal with those limitations and to automatize the software restructuring. MDD consists of the combination of generative programming, domain-specific languages and model transformations. It also aims to reduce the semantic gap between the program domain and the implementation, using high-level models that shield software developers from complexities of the underlying implementation platform~\cite{France:2007:MDC:1253532.1254709}.
		
%\section{Evaluation}
%	\input{tex/evaluation}
%	\subsection{Reuse Study Definition}
%		\input{tex/reuse_study_definition}
%	\subsection{Maintenance Study Definition}
%		\input{tex/maintenance_study_definition}
%	\subsection{Study Planning}
%		\input{tex/study_planning}
%	\subsection{Operation}
%		\input{tex/operation}
%	\subsection{Data Analysis and Interpretation}
%		\input{tex/data_analysis_and_interpretation}
%	\subsection{Hypotheses Testing}
%		\input{tex/hypotheses_testing}
%	\subsection{Threats to Validity}
%		\input{tex/threats_to_validity}
\section{Related Work}
		\input{tex/related_works}
\section{Conclusions}
		In this paper is presented the KnowDIME to support the refactoring of legacy systems based on ADM, which uses the KDM standard. It follows the theory of the horseshoe modernization model, which is threefold: (\textit{i}) \textbf{Reverser Engineering}, (\textit{ii}) \textbf{Reestructuring} and  (\textit{iii}) \textbf{Forward Engineering}. 

Firstly, all the artefacts of the legacy system must be transformed into PSMs by statically analyzing the legacy source code. Still in the first step, these PSMs are integrated into a KDM model, i.e., a PIM model, through a M2M transformation implemented using ATL. Secondly, the KnowDIME applies a set of model refactorings and model optimizations in this PIM. Afterwards, KnowDIME executes a set of M2M transformation taking as input the PIM and producing as output a model conforming to the KDM models into UML meta model. Finally, an improved system is obtained from this UML by means of a set of M2C transformation; additionally, the generated code can be complemented by the software engineer in accordance with the more detailed specifications of the application business rules, such as the implementation of specific behaviors and features not covered by the code generation.

We believe that KnowDIME makes a contribution to the challenges of Software Engineering which focuses on mechanisms to support the automation of software refactoring process. Long term future work involves conducting experiments to evaluate the level of maintenance provided by KnowDIME. It is worth highlighting that KnowDIME is open source and it can be downloaded at\textit{~www.dc.ufscar.br/$\sim$valter/crossfire}.


\section*{Acknowledgments}
	\input{tex/acknowledgment}






% trigger a \newpage just before the given reference
% number - used to balance the columns on the last page
% adjust value as needed - may need to be readjusted if
% the document is modified later
%\IEEEtriggeratref{8}
% The "triggered" command can be changed if desired:
%\IEEEtriggercmd{\enlargethispage{-5in}}

% references section

% can use a bibliography generated by BibTeX as a .bbl file
% BibTeX documentation can be easily obtained at:
% http://www.ctan.org/tex-archive/biblio/bibtex/contrib/doc/
% The IEEEtran BibTeX style support page is at:
% http://www.michaelshell.org/tex/ieeetran/bibtex/
\bibliographystyle{IEEEtran}
% argument is your BibTeX string definitions and bibliography database(s)
%\bibliography{IEEEabrv,IEEEexample}
\bibliography{IEEEabrv,revisaoln}
%
% <OR> manually copy in the resultant .bbl file
% set second argument of \begin to the number of references
% (used to reserve space for the reference number labels box)
%
%\begin{thebibliography}{1}
%
%\bibitem{IEEEhowto:kopka}
%H.~Kopka and P.~W. Daly, \emph{A Guide to \LaTeX}, 3rd~ed.\hskip 1em plus
%  0.5em minus 0.4em\relax Harlow, England: Addison-Wesley, 1999.
%
%\end{thebibliography}




% that's all folks
\end{document}


 